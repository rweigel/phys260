\newif\ifsolutions
\solutionstrue % Show solutions
%\solutionsfalse % Hide solutions

\documentclass{article}
\usepackage{geometry}
\geometry{margin=1in}
\usepackage{tikz}
\usepackage{amssymb}

% fleqn allows setting indent of display math
\usepackage[fleqn]{amsmath}
\setlength{\mathindent}{0pt} % Set indent
% Disable equation numbering (https://tex.stackexchange.com/a/360378)
\makeatletter
\renewcommand\tagform@[1]{}
\makeatother

% Allow Unicode (some, e.g., © and £ at least)
% https://tex.stackexchange.com/questions/370278/is-there-any-reason-to-use-inputenc
\usepackage[utf8]{inputenc}

% Hyperlinks
\usepackage{hyperref}
\hypersetup{colorlinks=true, urlcolor=blue, linkcolor=blue}

% Prevent indentation of paragraphs
\setlength\parindent{0pt}
\setlength{\parskip}{\baselineskip}

% Spacing above/below equations
% https://tex.stackexchange.com/a/69678
\AtBeginDocument{%
 \abovedisplayskip=-\parskip
 \abovedisplayshortskip=-\parskip
 \belowdisplayskip=0pt
 \belowdisplayshortskip=0pt
}

% Allow 3 additional subsection levels
% https://tex.stackexchange.com/a/60212
\usepackage{titlesec}
\setcounter{secnumdepth}{6}
% H4 in HTML
\titleformat{\paragraph}{\normalfont\normalsize\bfseries}{\theparagraph}{1em}{}
\titlespacing*{\paragraph}{0pt}{3.25ex plus 1ex minus .2ex}{1.5ex plus .2ex}
% H5 in HTML
\titleformat{\subparagraph}{\normalfont\normalsize\bfseries}{\thesubparagraph}{1em}{}
\titlespacing*{\subparagraph}{0pt}{3.25ex plus 1ex minus .2ex}{1.5ex plus .2ex}
% H6 in HTML
\titleformat{\subsubparagraph}{\normalfont\normalsize\bfseries}{\thesubsubparagraph}{1em}{}
\titlespacing*{\subsubparagraph}{0pt}{3.25ex plus 1ex minus .2ex}{1.5ex plus .2ex}

% So enumerate at all levels is numbers
% https://tex.stackexchange.com/questions/78842/nested-enumeration-numbering
\renewcommand{\labelenumii}{\arabic{enumii}.}
\renewcommand{\labelenumiii}{\arabic{enumiii}.}
\renewcommand{\labelenumiv}{\arabic{enumiv}.}

\renewcommand{\mbox}{\text}
\newcommand{\ds}[0]{\displaystyle}
\newcommand{\ihat}[0]{\hat{\boldsymbol{\imath}}}
\newcommand{\jhat}[0]{\hat{\boldsymbol{\jmath}}}
\newcommand{\khat}[0]{\hat{\boldsymbol{k}}}
\newcommand{\xhat}[0]{\hat{\mathbf{x}}}
\newcommand{\yhat}[0]{\hat{\mathbf{y}}}
\newcommand{\zhat}[0]{\hat{\mathbf{z}}}
\newcommand{\rhat}[0]{\hat{\mathbf{r}}}
\newcommand{\bfvec}[1]{\vec{\mathbf{#1}}}
\newcommand{\bfcdot}[0]{\boldsymbol{\cdot}}

\usepackage{fancyhdr}
\pagestyle{fancy}
\lhead{Sound Intensity Level}
\rhead{\thepage}
\fancyfoot{}

\begin{document}

\section{Introduction}

The intensity of a sound wave, $I$, is the average powerm $P$ per unit area from a sound wave that passes through or into a surface if area $A$,

$$I = \frac{P}{A}$$

$I$ has units of Power/Area, which has SI units of $(\text{Joules/s})/\text{m}^2=\text{Watts}/\text{m}^2$. The intensity of various sources of sound is shown in the third column of the following table from Young and Freedman, 14th Edition. The second column is discussed in section 3 of this activity.

\input{figures/Table16.2.png}

If the sound wave is created by a point source with power $P$, is emitted uniformly in all direction (isotropically), and there are no reflections or obstructions, the intensity varies with the distance $r$ from the point source according to

$$I = \frac{P}{4\pi r^2}$$

Because $I$ is proportional to $r^2$, if the disance from the source doubles, the intensity decreases by a factor of four.

\newpage

\section{Sound Intensity, $I$}

Suppose a bird sitting on the top of a lamp post emits sound with a power of $1\text{ W}$; assume that the relationship $I = P/4\pi r^2$ applies.

\begin{enumerate}

  \item What is $I$ at a point $1\text{ m}$ away?

        \ifsolutions
        \textbf{Answer}:

        $$I = \frac{1\text{ W}}{4\pi (1\text{ m})^2} = \frac{1}{4\pi} \frac{\text{ W}\phantom{^2}}{\text{m}^2}$$
        \else
        \vskip 120pt
        \fi

  \item What is $I$ at a point $10\text{ m}$ away?

        \ifsolutions
        \textbf{Answer}: If the distance increases by a factor of $10$, we expect $I$ to descrease a factor of $100$ (because $I$ is inversely proportional to area and area depends on the square of the distance). So

        $$I = \frac{1}{400\pi} \frac{\text{ W}\phantom{^2}}{\text{m}^2}$$
        \else
        \vskip 120pt
        \fi

\end{enumerate}

\vskip 0.75pt

A siren is emitting sound at a constant intensity level; assume that the relationship $I = P/4\pi r^2$ applies.

\begin{enumerate}

  \item[3.] If you move three times farther away from the siren, by what ratio does the sound intensity change?

            \ifsolutions
            \textbf{Answer}:

            Because intensity is proportional to $1/r^2$, moving 3 time farther away will decrease the intensity by a factor of $1/9$. This can be shown in more detail by noting that $I_1 = P_1/A_1$, $I_2=P_2/A_2$ and the power is constant, so $P_1=P_2$. Thus,

            $$\frac{I_2}{I_1} = \frac{P/A_2}{P/A_1} = \frac{4\pi d_1^2}{4\pi d_2^2} =  \frac{d_1^2}{d_2^2}$$

            If $I_1$ is your initial position at $d_1$, then $d_2 = 3d_1$ and

            $$\frac{I_2}{I_1} =  \frac{d_1^2}{d_2^2} = \frac{d_1^2}{(3d_1)^2} = \frac{1}{9}$$
            \else
            \vskip 120pt
            \fi

  \item[4.] If instead you moved four times closer to the siren, by what ratio does the sound intensity change?

            \ifsolutions
            \textbf{Answer}: Increases by a factor of $16$.

            $$\frac{I_2}{I_1} = \frac{d_1^2}{d_2^2} =  \frac{d_1^2}{(d_1/4)^2} = 16$$
            \else
            %\vspace{10em}
            \fi

\end{enumerate}

\newpage

\section{Sound Intensity Level, $\beta$}

Because of the sound intensity values of human hearing span a very large range, from $0.0000000000001$ to $100$ $\text{W}/\text{m}^2$ (see the table in the introduction), we define an alternative measure of intensity called the sound intensity \emph{level}, $\beta$. The equation that relates sound intensity level $I$, with sound intensity, $\beta$, is

$$\beta = (10 \text{ dB})\log_{10}\left(\frac{I}{I_o}\right)$$

where $\text{dB}$ stands for ``decibels".

One advantage to using this formula is that its values span a much smaller range, from 0--140, as shown in the second column of the table in the introduction. Another advantage is that the reference value of zero corresponds to something easily interpreted: $0\text{ dB}$ means something that is barely audible by humans.

\begin{enumerate}

  \item Logarithmic scales are often used in science an engineering. Give at least one other example besides sound intensity level that is a quantity that is based on a logarithmic scale.

        \ifsolutions
        \textbf{Answer}: pH scale for acidity, Richter earthquake magnitude scale.
        \else
        \vskip 96pt
        \fi

  \item If you increase the sound intensity of a speaker on a TV from $I_1$ to $I_2$ and $I_2/I_1=10$, what is $\beta_2-\beta_1$?

        \ifsolutions
        In the table, each factor of $10$ increase in $I$ corresponds to an change in $\beta$ by $+10$. So $\beta_2-\beta_1 = 10\text{ dB}$. We can also use

        $$\beta_1 = (10 \text{ dB})\log_{10}\left(\frac{I_1}{I_o}\right)$$

        $$\beta_2 = (10 \text{ dB})\log_{10}\left(\frac{I_2}{I_o}\right)$$

        $$\beta_2-\beta_1 = (10 \text{ dB})\left[\log_{10}\left(\frac{I_2}{I_o}\right)-\log_{10}\left(\frac{I_1}{I_o}\right)\right]$$

        Using $\log_{10}y - \log_{10}x = \log_{10}(y/x)$ gives

        $$\beta_2-\beta_1 = (10 \text{ dB})\left[\log_{10}\left(\frac{I_2}{I_1}\right)\right]$$

        So if $I_2/I_1 = 10$, we have

        $$\beta_2-\beta_1 = (10 \text{ dB})\left[\log_{10}(10)\right]= 10 \text{ dB}$$
        \else
        \vskip 96pt
        \fi

  \item If you increase the sound intensity level of a speaker on a TV from $\beta_1$ to $\beta_2$ and $\beta_2-\beta_1=20\text{ dB}$, what is $I_2/I_1$?

        \ifsolutions
        \textbf{Answer}: 

        $$20\text{ dB} = (10 \text{ dB})\left[\log_{10}\left(\frac{I_2}{I_1}\right)\right]$$

        Dividing both sides by $10 \text{ dB}$ gives

        $$2 = \log_{10}\left(\frac{I_2}{I_1}\right)$$

        Raising both sides by $10$ and using the identity $10^{\log_{10}x} = x$ gives

        $$10^2 = \frac{I_2}{I_1}$$

        This could also have been determined using the table in the introduction. The difference in $\beta$ for a whisper ($20\text{ dB}$) to $\beta$ for the threshold of hearing ($0\text{ dB}$) is $20\text{ dB}$. The ratio of $I$ for a whisper ($10^{-10}\text{ W}/\text{m}^2$) to $I$ for the threshold of hearing ($10^{-12}\text{ W}/\text{m}^2$) is $10^{-10}/10^{-12}=10^2$.
        \else
        \vskip 72pt
        \fi

\end{enumerate}

\vskip 0.75pt

\begin{enumerate}

  \item[4.] A city council adopted a law to reduce the maximum allowed sound intensity level of leaf blowers from $95\text{ dB}$ to $70\text{ dB}$. With the new law, what is the ratio of the new maximum allowed intensity to the previously allowed intensity?

            \ifsolutions
            \textbf{Answer}:

            From the solution to problem 3.2,

            $$\beta_2-\beta_1 = (10 \text{ dB})\left[\log_{10}\left(\frac{I_2}{I_1}\right)\right]$$

            If $\beta_2$ corresponds to the new maximum sound intensity level and $\beta_1$ the old,

            $$(70-95)\text{ dB} =  (10 \text{ dB})\log_{10}\left(\frac{I_2}{I_1}\right)$$

            $$-2.5 = \log_{10}\left(\frac{I_2}{I_1}\right)$$

            $$10^{-2.5} = \frac{I_2}{I_1}$$

            $$\frac{I_2}{I_1} = \frac{1}{10^{2.5}} \simeq \frac{1}{316}$$
            \else
            \vskip 96pt
            \fi

\end{enumerate}

\vskip 0.75pt

A siren is emitting sound at a constant intensity level; assume that the relationship $I = P/4\pi r^2$ applies.

\begin{enumerate}

  \item[5.] If you move three times farther away from the siren, what is the change in the sound intensity level?

            \ifsolutions
            \textbf{Answer}: From problem 3.2,

            $$\beta_2-\beta_1 = (10 \text{ dB})\left[\log_{10}\left(\frac{I_2}{I_1}\right)\right]$$

            If you three times closer, $I_2/I_1=1/9$, so

            $$\beta_2-\beta_1 = (10 \text{ dB})\log_{10}(1/9) \simeq -9.5\text{ dB}$$
            \else
            \vskip 96pt
            \fi

  \item[6.] If instead you moved four times closer to the siren, what is the change in the sound intensity level ?

            \ifsolutions
            \textbf{Answer}:

            $$\beta_2-\beta_1 = (10 \text{ dB})\log_{10}(16) \simeq 12\text{ dB}$$
            \else
            \vskip 96pt
            \fi

\end{enumerate}

\vskip 0.75pt

\begin{enumerate}

  \item[7.] You are trying to hear a juicy conversation, but from your distance of 15.0 m the sound intensity level is that of only an average whisper, $20.0 \text{ dB}$. How close should you move for the sound intensity level to the same as a quiet radio in your home? Show your work.

            \ifsolutions
            From the solution to problem 3.2,

            $$\beta_2-\beta_1 = (10 \text{ dB})\left[\log_{10}\left(\frac{I_2}{I_1}\right)\right]$$

            From the solution to problem 2.2,

            $$\frac{I_2}{I_1} = \frac{d_1^2}{d_2^2}$$

            $$\beta_2-\beta_1 = (10 \text{ dB})\left[\log_{10}\left(\frac{d_1^2}{d_2^2}\right)\right]$$

            From the table in the introduction the sound intensity level for a quiet radio in your home is $40.0\text{ dB}$. Let $\beta_2=40.0\text{ dB}$ be the level after you move to $d_2$.

            $$(40-20)\text{ dB} = (10 \text{ dB})\left[\log_{10}\left(\frac{d_1^2}{d_2^2}\right)\right]$$

            $$2 = \left[\log_{10}\left(\frac{d_1^2}{d_2^2}\right)\right]$$

            $$10^2 = \frac{d_1^2}{d_2^2}$$

            $\ds10 = \frac{d_1}{d_2} \Rightarrow d_2 = d_1/10 = (15 \text{ m})/100 = 1.5\text{ m}$
            \else
            \vskip 96pt
            \fi

  \item[8.] Solve for $I$ in the equation $\beta = (10 \text{ dB})\log_{10}\left(\frac{I}{I_o}\right)$

            \ifsolutions
            \textbf{Answer}:

            Divide both sides by ${10 \text{ dB}}$

            $$\beta/(10 \text{ dB})  = \log_{10}\left(\frac{I}{I_o}\right)$$

            Raising both sides to the power of $10$ gives

            $$10^{\beta/(10 \text{ dB})} = 10^{\log_{10}\left(\frac{I}{I_o}\right)}$$

            Using the identity $10^{\log_{10}x} = x$,

            $$10^{\beta/(10 \text{ dB})} = \frac{I}{I_o}$$

            Solving for $I$ gives   

            $I=I_o 10^{\beta/(10\text{ dB})}$
            \else
            \vskip 96pt
            \fi

  \item[9.] Solve for $I_2/I_1$ in the equation $\beta_2-\beta_1 = (10 \text{ dB})\log_{10}\left(\frac{I_2}{I_1}\right)$

            \ifsolutions
            \textbf{Answer}:

            $$\frac{I_2}{I_1}=10^{(\beta_2-\beta_1)/(10\text{ dB})}$$
            \else
            \vskip 96pt
            \fi

\end{enumerate}

\end{document}