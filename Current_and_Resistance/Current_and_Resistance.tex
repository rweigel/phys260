\newif\ifsolutions
\solutionstrue % Show solutions
%\solutionsfalse % Hide solutions

\documentclass{article}
\usepackage{geometry}
\geometry{margin=1in}
\usepackage{tikz}
\usepackage{amssymb}

% fleqn allows setting indent of display math
\usepackage[fleqn]{amsmath}
\setlength{\mathindent}{0pt} % Set indent
% Disable equation numbering (https://tex.stackexchange.com/a/360378)
\makeatletter
\renewcommand\tagform@[1]{}
\makeatother

% Allow Unicode (some, e.g., © and £ at least)
% https://tex.stackexchange.com/questions/370278/is-there-any-reason-to-use-inputenc
\usepackage[utf8]{inputenc}

% Hyperlinks
\usepackage{hyperref}
\hypersetup{colorlinks=true, urlcolor=blue, linkcolor=blue}

% Prevent indentation of paragraphs
\setlength\parindent{0pt}
\setlength{\parskip}{\baselineskip}

% Spacing above/below equations
% https://tex.stackexchange.com/a/69678
\AtBeginDocument{%
 \abovedisplayskip=-\parskip
 \abovedisplayshortskip=-\parskip
 \belowdisplayskip=0pt
 \belowdisplayshortskip=0pt
}

% Allow 3 additional subsection levels
% https://tex.stackexchange.com/a/60212
\usepackage{titlesec}
\setcounter{secnumdepth}{6}
% H4 in HTML
\titleformat{\paragraph}{\normalfont\normalsize\bfseries}{\theparagraph}{1em}{}
\titlespacing*{\paragraph}{0pt}{3.25ex plus 1ex minus .2ex}{1.5ex plus .2ex}
% H5 in HTML
\titleformat{\subparagraph}{\normalfont\normalsize\bfseries}{\thesubparagraph}{1em}{}
\titlespacing*{\subparagraph}{0pt}{3.25ex plus 1ex minus .2ex}{1.5ex plus .2ex}
% H6 in HTML
\titleformat{\subsubparagraph}{\normalfont\normalsize\bfseries}{\thesubsubparagraph}{1em}{}
\titlespacing*{\subsubparagraph}{0pt}{3.25ex plus 1ex minus .2ex}{1.5ex plus .2ex}

% So enumerate at all levels is numbers
% https://tex.stackexchange.com/questions/78842/nested-enumeration-numbering
\renewcommand{\labelenumii}{\arabic{enumii}.}
\renewcommand{\labelenumiii}{\arabic{enumiii}.}
\renewcommand{\labelenumiv}{\arabic{enumiv}.}

\renewcommand{\mbox}{\text}
\newcommand{\ds}[0]{\displaystyle}
\newcommand{\ihat}[0]{\hat{\boldsymbol{\imath}}}
\newcommand{\jhat}[0]{\hat{\boldsymbol{\jmath}}}
\newcommand{\khat}[0]{\hat{\boldsymbol{k}}}
\newcommand{\xhat}[0]{\hat{\mathbf{x}}}
\newcommand{\yhat}[0]{\hat{\mathbf{y}}}
\newcommand{\zhat}[0]{\hat{\mathbf{z}}}
\newcommand{\rhat}[0]{\hat{\mathbf{r}}}
\newcommand{\bfvec}[1]{\vec{\mathbf{#1}}}
\newcommand{\bfcdot}[0]{\boldsymbol{\cdot}}

\begin{document}

\section{Introduction}

\subsection{Definitions}

The electric current in a wire is defined as 

$$I = \frac{dQ}{dt}$$

where $dQ$ is the total amount of charge that passes through a cross--section of the wire in a differential amount of time, $dt$.

If $q$ is the charge (in Coulombs) of each flowing charge, $n$ is their number per volume (``number density"), $v_d$ their average speed along the wire (called the ``drift speed"), then

$I = n|q|v_dA$, where $A$ is the cross--sectional area of the wire.

The average current density is defined as

$\ds J=\frac{I}{A}$, which can be written as $J = n|q|v_d$.

\subsection{Ohm's Law}

If an electric field exists in a wire (by, for example, connecting its ends to a battery), the charges will accelerate until they collide with another particle and decelerate (collisions resist the flow). The net result will be current -- a flow of charges with an average drift speed. Experimentally, it has been shown that in many materials, the electric field is proportional to the current density: 

$$E = \rho J$$

where the value of the proportionality constant $\rho$, called resistivity, depends on the material. This is one version of Ohm's law.

For a wire of length $L$ and constant cross--sectional area $A$, Ohm's law can also be written as $V = ({\rho L}/{A}) I$. If we define resistance as $R = {\rho L}/{A}$, then we have another relationship that is also referred to as Ohm's law: 

$V = I R$.

In this form, the interpretation is that the voltage between the ends of a wire is proportional to the current in the wire, with the proportionality constant of $R$.

\newpage

\section{Problem I -- Definitions and Ohm's Law}

A $9$--volt power source is connected to a wire of length $10$ meters with a circular cross--section and radius of $0.01$ meters. The wire has a resistivity of $10^{-8}\text{ }\Omega\cdot\text{m}$. The number density of flowing charges is $10^{28}/\text{m}^3$. Assume Ohm's law applies and the charges that flow are electrons.

\begin{enumerate}

  \item What is the resistance (with units) of the wire?

        \ifsolutions
        \textbf{Answer}: $\ds R=\frac{\rho L}{A} = \frac{(10^{-8}\text{ }\Omega\cdot\text{m})(10\text{ m})}{\pi (0.01\text{ m})^2} = \frac{10^{-3}}{\pi}\text{ }\Omega$. When solving circuit problems, the resistors involved typically have a much larger than this, which is why we neglect the resistance of the wires.
        \else
        \vskip 84pt
        \fi

  \item What is the current (with units) in the wire?

        \ifsolutions
        \textbf{Answer}: $\ds I = \frac{9\text{ v}}{\frac{10^{-3}}{\pi}\text{ }\Omega}=9,000\pi\text{ A}$. This is a huge current. If you look at the back of an electronic device, you will see a rating on the order of $1\text{ A}$. Household circuit breakers are set to open (``trip") at approximately $15\text{ A}$. Most power sources cannot supply current at this rate; even if the power source could supply this current, the amount of heat created would lead to a fire or melting of the wire.
        \else
        \vskip 84pt
        \fi

  \item How much charge (in Coulombs) flows past a cross--section of the wire per second?

        \ifsolutions
        \textbf{Answer}: $9,000\pi\text{ C}$
        \else
        \vskip 84pt
        \fi

  \item How many electrons flow past a cross--section of the wire per second?

  \item What is the current density (with units)?

        \ifsolutions
        \textbf{Answer}: $\ds J=\frac{I}{A} = 9\cdot 10^7 \frac{\text{A}}{\text{m}^2}$
        \else
        \vskip 84pt
        \fi

  \item What is the drift velocity of electrons in the wire? (The charge on an electron is $-1.6·10^{-19}\text{ C}$.)

        \ifsolutions
        \textbf{Answer}: $v_d = J/n|q| \simeq 6\text{ cm/s}$
        \else
        \vskip 84pt
        \fi

  \item Based on the description of how charged particles flow in a wire, explain why the resistance of a cylindrical wire is proportional to its length and inversely proportional to the square of its radius.

        \ifsolutions
        \textbf{Answer}: See the description in your textbook.
        \else

        \fi

\end{enumerate}

\ifsolutions

\else

\newpage
\fi

\section{Problem II -- Definitions and Ohm's Law}

A power source is connected to a $20$ meter long wire with a circular cross--section and radius of $0.01$ meters. The wire has a resistivity of $10^{-7}\text{ }\Omega\cdot\text{m}$. The number density of flowing charges is $10^{27}/\text{m}^3$. The current in the wire was measured and found to be $1\text{ A}$. Assume Ohm's law applies and the charges that flow are electrons.

\begin{enumerate}

  \item What is the resistance (with units) of the wire?

        \ifsolutions
        \textbf{Answer}: $(.02/\pi)\text{ }\Omega$
        \else
        \vskip 84pt
        \fi

  \item How much charge (with units) flows past a cross--section of the wire per second?

        \ifsolutions
        \textbf{Answer}: $1\text{ C}$
        \else
        \vskip 84pt
        \fi

  \item What is the current density (with units)?

        \ifsolutions
        \textbf{Answer}: $\ds J = \frac{10^4}{\pi}\frac{\text{A}}{\text{m}^2}$
        \else
        \vskip 84pt
        \fi

  \item What is the drift velocity of electrons in the wire? (The charge on an electron is $-1.6·10^{-19}\text{ C}$.)

        \ifsolutions
        \textbf{Answer}: $v_d = J/n|q| = 0.02\text{ mm/s}$
        \else
        \vskip 84pt
        \fi

\end{enumerate}

\ifsolutions

\else

\newpage
\fi

\section{Problem III -- Current Through a Cylindrical Shell}

If a cylindrical wire with an inner radius $a$ and outer radius $b$ carries a current $I$, what is $J$?

\ifsolutions
\textbf{Answer}: $\ds J=\frac{I}{\pi(b^2-a^2)}$
\else
\vskip 84pt
\fi

\section{Problem IV -- $I = n|q|v_dA$ derivation}

Derive the relationship $I = n|q|v_dA$. Provide a diagram.

\ifsolutions
\textbf{Answer}: See textbook.
\else

\fi

\end{document}