\hypertarget{overview}{%
\section{Overview}\label{overview}}

This activity covers topics in
\href{https://drive.google.com/file/d/1JS_pBuNEwXdz9IzpSBFPJffgVacZmqN7/view?usp=sharing_remove_}{Section
21.3-4 of Young and Freedman 2015, 14th Edition}.

\textbf{Electric Force}

Coulomb's Law in compact form is

\[\mathbf{F}_{1\mbox{ on } 2}=kq_1q_2\frac{\hat{\mathbf{r}}}{r^2}\]

where \(\hat{\mathbf{r}}\) is the unit vector that points from the
position of \(q_1\) to \(q_2\) and \(r\) is the distance between \(q_1\)
and \(q_2\).

\textbf{Electric Field}

The electric field vector, \(\mathbf{E}\), is a quantity that we assign
to a point in space. Given this quantity, we can compute the force on a
charge \(Q\) will experience if it is placed at that point in space
using the equation \(\mathbf{F}=Q\mathbf{E}\).

To find \(\mathbf{E}\) at any point in space, compute the force
\(\mathbf{F}\) due to all other charges on a hypothetical (or ``test'')
charge \(q_o\) at a point where you want to know \(\mathbf{E}\). To find
\(\mathbf{E}\) at that point, divide \(\mathbf{F}\) by \(q_o\).

\[\mathbf{E} = \frac{\mathbf{F}}{q_o}\]

\%To find \(\mathbf{F}\) when a different charge \(Q\) is placed where
\(q\) was, multiply \(\mathbf{E}\) by \(Q\).

\hypertarget{example}{%
\section{Example}\label{example}}

Charge \(q_1\) is at \((x,y)=(-a,-a)\).

Find the electric field at \((x,y)=(a,a)\) in the form
\(\mathbf{E}=E_x\xhat + E_y\yhat\).

\textbf{Solution}

According to the prescription given, to find the electric field at a
point in space, we put a hypothetical ``test'' charge \(q_o\) at that
point, compute the force on it due to all other charges, and then use

\[\mathbf{E} = \frac{\mathbf{F}}{q_o}\]

The force a charge \(q_1\) at \((x,y)=(-a,-a)\) exerts on a charge
\(q_2\) at \((x,y)=(a,a)\) was computed in a previous activity. We can
use the answer after the replacement of \(q_2\) with \(q_o\). The result
is

\(\mathbf{F}=k\frac{q_1q_o}{8a^2}\left[\frac{1}{\sqrt{2}}\xhat + \frac{1}{\sqrt{2}}\yhat\right]\).
The electric field is then
\(\mathbf{E} = \frac{\mathbf{F}}{q_o} = k\frac{q_1}{8a^2}\left[\frac{1}{\sqrt{2}}\xhat + \frac{1}{\sqrt{2}}\yhat\right]\)

\newpage

\hypertarget{problem}{%
\section{Problem}\label{problem}}

In the previous example, there was only one charge responsible for
creating the electric field \(\mathbf{E}\). To find the electric field
when there are more charges, superposition can be used.

Charge \(q_1 = +q\) is at \((x, y) = (a, 0)\), charge \(q_2 = +q\) is at
\((x, y) = (-a, 0)\), and charge \(q_3 = -q\) is at \((x, y) = (0, a)\).
Assume that the quantity associated with \(q\) is positive.

\begin{enumerate}
\def\labelenumi{\arabic{enumi}.}
\tightlist
\item
  Draw this charge configuration below.
\end{enumerate}

\begin{enumerate}
\def\labelenumi{\arabic{enumi}.}
\setcounter{enumi}{1}
\tightlist
\item
  Why does it not make sense to ask what the electric force is at the
  origin?
\end{enumerate}

In the following,

\begin{enumerate}
\def\labelenumi{\arabic{enumi}.}
\setcounter{enumi}{2}
\tightlist
\item
  Find the electric field at the origin due to \(q_1\). Write your
  answer in the form \(\mathbf{E}\_1=E_{x1}\xhat + E_{y1}\yhat\).
\end{enumerate}

\begin{enumerate}
\def\labelenumi{\arabic{enumi}.}
\setcounter{enumi}{3}
\tightlist
\item
  Find the electric field at the origin due to \(q_2\). Write your
  answer in the form \(\mathbf{E}\_2=E_{x2}\xhat + E_{y2}\yhat\).
\end{enumerate}

\begin{enumerate}
\def\labelenumi{\arabic{enumi}.}
\setcounter{enumi}{4}
\tightlist
\item
  Find the electric field at the origin due to \(q_3\). Write your
  answer in the form \(\mathbf{E}\_3=E_{x3}\xhat + E_{y3}\yhat\).
\end{enumerate}

\begin{enumerate}
\def\labelenumi{\arabic{enumi}.}
\setcounter{enumi}{5}
\tightlist
\item
  Find the electric field at the origin. Write your answer in the form
  \(\mathbf{E}=E_{x}\xhat + E_{y}\yhat\).
\end{enumerate}

\begin{enumerate}
\def\labelenumi{\arabic{enumi}.}
\setcounter{enumi}{6}
\tightlist
\item
  Will your answers to 3.-6. change if the problem had asked for the
  electric field at a different position? If so, which answers?
\end{enumerate}

\newpage

\begin{enumerate}
\def\labelenumi{\arabic{enumi}.}
\setcounter{enumi}{7}
\tightlist
\item
  Find the electric field at the origin if charge \(q_1=2q\) (instead of
  \(q\)).
\end{enumerate}

\begin{enumerate}
\def\labelenumi{\arabic{enumi}.}
\setcounter{enumi}{8}
\tightlist
\item
  Find the electric field at the origin if charge \(q_1=-2q\) (instead
  of \(q\)).
\end{enumerate}
