\newif\ifsolutions
\solutionstrue % Show solutions
%\solutionsfalse % Hide solutions

\documentclass{article}
\usepackage{geometry}
\geometry{margin=1in}
\usepackage{tikz}
\usepackage{amssymb}

% fleqn allows setting indent of display math
\usepackage[fleqn]{amsmath}
\setlength{\mathindent}{0pt} % Set indent
% Disable equation numbering (https://tex.stackexchange.com/a/360378)
\makeatletter
\renewcommand\tagform@[1]{}
\makeatother

% Allow Unicode (some, e.g., © and £ at least)
% https://tex.stackexchange.com/questions/370278/is-there-any-reason-to-use-inputenc
\usepackage[utf8]{inputenc}

% Hyperlinks
\usepackage{hyperref}
\hypersetup{colorlinks=true, urlcolor=blue, linkcolor=blue}

% Prevent indentation of paragraphs
\setlength\parindent{0pt}
\setlength{\parskip}{\baselineskip}

% Spacing above/below equations
% https://tex.stackexchange.com/a/69678
\AtBeginDocument{%
 \abovedisplayskip=-\parskip
 \abovedisplayshortskip=-\parskip
 \belowdisplayskip=0pt
 \belowdisplayshortskip=0pt
}

% Allow 3 additional subsection levels
% https://tex.stackexchange.com/a/60212
\usepackage{titlesec}
\setcounter{secnumdepth}{6}
% H4 in HTML
\titleformat{\paragraph}{\normalfont\normalsize\bfseries}{\theparagraph}{1em}{}
\titlespacing*{\paragraph}{0pt}{3.25ex plus 1ex minus .2ex}{1.5ex plus .2ex}
% H5 in HTML
\titleformat{\subparagraph}{\normalfont\normalsize\bfseries}{\thesubparagraph}{1em}{}
\titlespacing*{\subparagraph}{0pt}{3.25ex plus 1ex minus .2ex}{1.5ex plus .2ex}
% H6 in HTML
\titleformat{\subsubparagraph}{\normalfont\normalsize\bfseries}{\thesubsubparagraph}{1em}{}
\titlespacing*{\subsubparagraph}{0pt}{3.25ex plus 1ex minus .2ex}{1.5ex plus .2ex}

% So enumerate at all levels is numbers
% https://tex.stackexchange.com/questions/78842/nested-enumeration-numbering
\renewcommand{\labelenumii}{\arabic{enumii}.}
\renewcommand{\labelenumiii}{\arabic{enumiii}.}
\renewcommand{\labelenumiv}{\arabic{enumiv}.}

\renewcommand{\mbox}{\text}
\newcommand{\ds}[0]{\displaystyle}
\newcommand{\ihat}[0]{\hat{\boldsymbol{\imath}}}
\newcommand{\jhat}[0]{\hat{\boldsymbol{\jmath}}}
\newcommand{\khat}[0]{\hat{\boldsymbol{k}}}
\newcommand{\xhat}[0]{\hat{\mathbf{x}}}
\newcommand{\yhat}[0]{\hat{\mathbf{y}}}
\newcommand{\zhat}[0]{\hat{\mathbf{z}}}
\newcommand{\rhat}[0]{\hat{\mathbf{r}}}
\newcommand{\bfvec}[1]{\vec{\mathbf{#1}}}
\newcommand{\bfcdot}[0]{\boldsymbol{\cdot}}

\usepackage{fancyhdr}
\pagestyle{fancy}
\lhead{Logarithms}
\rhead{\thepage}
\fancyfoot{}

\begin{document}

\section{Introduction -- Base $10$ Logarithms}

The motivation for the base 10 logarithm is that it reduces numbers raised by a power of 10 to the power the number was raised to. So $10^2$ becomes $2$, $10^3$ becomes $3$, etc. The base 10 logarithm is sometimes called the ``common logarithm".

In mathematical notation,

$\log_{10}(10^x) = x$

For example, $\log_{10}(10^{-5}) = -5$ and $\log_{10}(10^7) = 7$

(To take the base 10 logarithm of a number that is not exactly a power of $10$, use a calculator.)

Several identities follow as a result:

\begin{enumerate}

  \item If you raise a base 10 logged number by $10$, you get back the number that was logged.

        $10^{\log_{10}(x)} = x$

        For example,

        $10^{\log_{10}(7)} = 7$ and  $10^{\log_{10}(8.8)} = 8.8$

  \item The sum of two logged numbers is the log of the product of the numbers:

        $\log_{10}y + \log_{10}x = \log_{10}(yx)$;

        For example,

        $\log_{10}10 + \log_{10}100 = \log_{10}10\cdot 100 = \log_{10}10^3 = 3$

  \item The difference between two logged number is the log of the ratio of the numbers:

        $\log_{10}y - \log_{10}x = \log_{10}(y/x)$

        For example,

        $\log_{10}10 - \log_{10}100 = \log_{10}(10/100) = \log_{10}10^{-1} = -1$

\end{enumerate}

\subsection{Problems}

\begin{enumerate}

  \item What is $\log_{10}(0.000000001)$?

        \ifsolutions
        \textbf{Answer}: $\log_{10}(0.000000001)=\log_{10}(10^{-9})=-9$
        \else
        \vskip 24pt
        \fi

  \item What is $\log_{10}(10,000)$?

        \ifsolutions
        \textbf{Answer}: $\log_{10}(10,000)=\log_{10}(10^{4})=4$
        \else
        \vskip 24pt
        \fi

  \item $\log_{10}(10,000)+\log_{10}(0.000000001)=\log_{10}(x)$. Find $x$.

        \ifsolutions
        \textbf{Answer}: $\log_{10}(10^{4}) + \log_{10}(10^{-9}) = \log_{10}(10^{4}\cdot 10^{-9}) = \log_{10}(10^{-5})\Rightarrow x = 10^{-9}$
        \else
        \vskip 24pt
        \fi

  \item $\log_{10}(10,000)-\log_{10}(0.000000001)=\log_{10}(x)$. Find $x$.

        \ifsolutions
        \textbf{Answer}: $\log_{10}(10^{4}) - \log_{10}(10^{-9}) = \log_{10}(10^{4}/10^{-9}) = \log_{10}(10^{13})\Rightarrow x = 10^{13}$
        \else
        \vskip 24pt
        \fi

  \item If $x = x_o\log_{10}(y/y_o)$, solve for $y$.

        \ifsolutions
        \textbf{Answer}: $x/x_o = \log_{10}(y/y_o)$; $10^{x/x_o} = 10^{\log_{10}(y/y_o)}=y/y_o \Rightarrow y = y_o10^{x/x_o}$
        \else
        \vskip 24pt
        \fi

\end{enumerate}

\section{Introduction -- Base $e$ Logarithm}

The base $10$ logarithm reduces numbers raised by a power of 10 to the power the number was raised to.

The base $e$ logarithm reduces numbers raised by a power of $e$ to the power the number was raised to. It is represented by $\log_{e}x$, or more commonly, $\ln(x)$.

``$\ln$" represents the ``natural logarithm". The term ``natural" is used because the exponential $e$ appears in many natural problems, for example, some populations grow in proportion to $e^{t/\tau}$, where the constant $\tau$ is a growth rate.

In mathematical notation, $\ln(e^x) = x$; for example $\ln(e^{-5}) = -5$ and $\ln(e^7) = 7$

Several identities follow as a result:

\begin{enumerate}

  \item If you raise a base-$e$ logged number by $e$, you get back the number that was logged.

        $e^{\ln(x)} = x$

  \item The sum of two logged numbers is the log of the product of the numbers:

        $\ln(y) + \ln(x) = \ln(yx)$;

  \item The difference between two logged number is the log of the ratio of the numbers:

        $\ln(y) - \ln(x) = \ln(y/x)$

\end{enumerate}

\subsection{Problems}

\begin{enumerate}

  \item What is $\ln(e^3)$?

        \ifsolutions
        \textbf{Answer}: $3$
        \else
        \vskip 24pt
        \fi

  \item What is $\ln(1/e^3)$?

        \ifsolutions
        \textbf{Answer}: $-3$
        \else
        \vskip 24pt
        \fi

  \item $\ln(e^{-4})+\ln(e^{3}) = \ln(x)$. Find $x$

        \ifsolutions
        \textbf{Answer}: $\ln(e^{-4})+\ln(e^{3})=\ln(e^{-4}\cdot e^{3})=\ln(e^{-1})=-1$
        \else
        \vskip 24pt
        \fi

  \item $\ln(e^{-4})-\ln(e^{3}) = \ln(x)$. Find $x$.

        \ifsolutions
        \textbf{Answer}: $\ln(e^{-4})-\ln(e^{3})=\ln(e^{-4}/e^{3})=\ln(e^{-7})=-7$
        \else
        \vskip 24pt
        \fi

  \item If $x = x_o\ln(y/y_o)$, solve for $y$ in terms of $x$.

        \ifsolutions
        \textbf{Answer}: $x/x_o = \ln(y/y_o)$; $e^{x/x_o} = e^{\ln(y/y_o)}=y/y_o \Rightarrow y = y_oe^{x/x_o}$
        \else
        \vskip 24pt
        \fi

\end{enumerate}

\end{document}