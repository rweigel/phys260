\newif\ifsolutions
\solutionstrue % Show solutions
%\solutionsfalse % Hide solutions

\documentclass{article}
\usepackage{geometry}
\geometry{margin=1in}
\usepackage{tikz}
\usepackage{amssymb}

% fleqn allows setting indent of display math
\usepackage[fleqn]{amsmath}
\setlength{\mathindent}{0pt} % Set indent
% Disable equation numbering (https://tex.stackexchange.com/a/360378)
\makeatletter
\renewcommand\tagform@[1]{}
\makeatother

% Allow Unicode (some, e.g., © and £ at least)
% https://tex.stackexchange.com/questions/370278/is-there-any-reason-to-use-inputenc
\usepackage[utf8]{inputenc}

% Hyperlinks
\usepackage{hyperref}
\hypersetup{colorlinks=true, urlcolor=blue, linkcolor=blue}

% Prevent indentation of paragraphs
\setlength\parindent{0pt}
\setlength{\parskip}{\baselineskip}

% Spacing above/below equations
% https://tex.stackexchange.com/a/69678
\AtBeginDocument{%
 \abovedisplayskip=-\parskip
 \abovedisplayshortskip=-\parskip
 \belowdisplayskip=0pt
 \belowdisplayshortskip=0pt
}

% Allow 3 additional subsection levels
% https://tex.stackexchange.com/a/60212
\usepackage{titlesec}
\setcounter{secnumdepth}{6}
% H4 in HTML
\titleformat{\paragraph}{\normalfont\normalsize\bfseries}{\theparagraph}{1em}{}
\titlespacing*{\paragraph}{0pt}{3.25ex plus 1ex minus .2ex}{1.5ex plus .2ex}
% H5 in HTML
\titleformat{\subparagraph}{\normalfont\normalsize\bfseries}{\thesubparagraph}{1em}{}
\titlespacing*{\subparagraph}{0pt}{3.25ex plus 1ex minus .2ex}{1.5ex plus .2ex}
% H6 in HTML
\titleformat{\subsubparagraph}{\normalfont\normalsize\bfseries}{\thesubsubparagraph}{1em}{}
\titlespacing*{\subsubparagraph}{0pt}{3.25ex plus 1ex minus .2ex}{1.5ex plus .2ex}

% So enumerate at all levels is numbers
% https://tex.stackexchange.com/questions/78842/nested-enumeration-numbering
\renewcommand{\labelenumii}{\arabic{enumii}.}
\renewcommand{\labelenumiii}{\arabic{enumiii}.}
\renewcommand{\labelenumiv}{\arabic{enumiv}.}

\renewcommand{\mbox}{\text}
\newcommand{\ds}[0]{\displaystyle}
\newcommand{\ihat}[0]{\hat{\boldsymbol{\imath}}}
\newcommand{\jhat}[0]{\hat{\boldsymbol{\jmath}}}
\newcommand{\khat}[0]{\hat{\boldsymbol{k}}}
\newcommand{\xhat}[0]{\hat{\mathbf{x}}}
\newcommand{\yhat}[0]{\hat{\mathbf{y}}}
\newcommand{\zhat}[0]{\hat{\mathbf{z}}}
\newcommand{\rhat}[0]{\hat{\mathbf{r}}}
\newcommand{\bfvec}[1]{\vec{\mathbf{#1}}}
\newcommand{\bfcdot}[0]{\boldsymbol{\cdot}}

\usepackage{fancyhdr}
\pagestyle{fancy}
\lhead{Magnetic Field of a Moving Charge}
\rhead{\thepage}
\fancyfoot{}

\begin{document}

\section{Introduction}

In previous activities, you computed the force on moving charges in a region of space where there is a magnetic field. No mention was made of how the magnetic field was created.

In this activity, you compute the magnetic field created by moving charges.

The magnetic field due to a point charge $q$ moving with velocity $\bfvec{v}$ (when $|\bfvec{v}|$ is small compared to the speed of light) is

$$
\bfvec{B} = \frac{\mu_o}{4\pi}\frac{q\bfvec{v}\times\hat{\mathbf{r}}}{r^2}
$$

where $\rhat$ is the unit vector that points from the position of $q$ to the point in space where we want to know $\bfvec{B}$, and $r$ is the distance between $q$ and that point.

To find $\rhat$ (see also the $\rhat$ Unit Vector activity), 

\begin{enumerate}

  \item draw a vector, $\bfvec{r}$, from $q$ to the point in space where you want to know $\bfvec{B}$;

  \item Write $\bfvec{r}$ in the form $\bfvec{r}=r_x\ihat+r_y\jhat$; then

  \item $\rhat=\bfvec{r}/r$, where $r=\sqrt{r_x^2+r_y^2}$.

\end{enumerate}

In this activity, the examples and solutions are given using the above approach for computing $\bfvec{B}$. An alternative is to use the fact that $\bfvec{v}\times\hat{\mathbf{r}}=|\bfvec{v}|\sin\phi=v\sin\phi$, where $\phi$ is the angle between $\bfvec{v}$ and $\hat{\mathbf{r}}$ and $0 \le\phi \le 180^{\circ}$. With this, the magnitude of the magnetic field is

$$
B = \frac{\mu_o}{4\pi}\frac{|q|v\sin\phi}{r^2}
$$

and the right--hand rule can be used to determine the direction of $\bfvec{B}$. See the Cross Products activity for a discussion of when and how to compute the cross-product using this method.

\ifsolutions

\else

\newpage
\fi

\section{Example}

If $q$ is at $(x,y)=(-a,-a)$ and has a velocity of $\bfvec{v}=v_o\ihat$, find $\bfvec{B}$ at $(x,y)=(a,a)$.

\textbf{Solution}

To find $\hat{\mathbf{r}}$, we draw a vector from $q$ to the point where we want to compute $\bfvec{B}$.

\input{figures/Example.tikz}

Based on the diagram, $\bfvec{r}=2a\ihat + 2a\jhat$ and $r = \sqrt{(2a)^2+(2a)^2}=2\sqrt{2}a$, so

$$\hat{\mathbf{r}}=\frac{\bfvec{r}}{r} = \left[\frac{1}{\sqrt{2}}\ihat + \frac{1}{\sqrt{2}}\jhat\right]$$

The cross-product is

$$\bfvec{v}\times\hat{\mathbf{r}} = v_o\ihat\times\left[\frac{1}{\sqrt{2}}\ihat + \frac{1}{\sqrt{2}}\jhat\right] = \frac{v_o}{\sqrt{2}}(\ihat\times\jhat) = \frac{v_o}{\sqrt{2}}\khat$$

Substitution into 

$$\bfvec{B} = \frac{\mu_o}{4\pi}\frac{q\bfvec{v}\times\hat{\mathbf{r}}}{r^2}$$

gives

$$\bfvec{B}(a,a) = \frac{\mu_o}{4\pi} \frac{q\frac{v_o}{\sqrt{2}}\khat}{(2\sqrt{2}a)^2} = \frac{\mu_o}{4\pi} \frac{qv_o}{(8\sqrt{2})a^2}\khat$$

Check: Use the right--hand rule for cross products on $\bfvec{v}\times\hat{\mathbf{r}}$ to verify that the result is out of the page. (Why do we know that the $\khat$ direction is out of the page?)

\ifsolutions

\else

\newpage
\fi

\section{Problem I}

If $q$ is at $(x,y)=(a,a)$ and has a velocity of $\bfvec{v}=v_o\ihat$, find $\bfvec{B}$ at $(x,y)=(-a,-a)$.

\ifsolutions
\textbf{Solution}

$\bfvec{r}=-2a\ihat - 2a\jhat$ and $r=2\sqrt{2}a$, so

$$\hat{\mathbf{r}}=\frac{\bfvec{r}}{r} = \left[-\frac{1}{\sqrt{2}}\ihat - \frac{1}{\sqrt{2}}\jhat\right]$$

The cross--product is

$$\bfvec{v}\times\hat{\mathbf{r}}=v_o\ihat\times\left[-\frac{1}{\sqrt{2}}\ihat - \frac{1}{\sqrt{2}}\jhat\right] = -\frac{v_o}{\sqrt{2}}(\ihat\times\jhat) = -\frac{v_o}{\sqrt{2}}\khat$$

Substitution into 

$$\bfvec{B} = \frac{\mu_o}{4\pi}\frac{q\bfvec{v}\times\hat{\mathbf{r}}}{r^2}$$

gives

$$\bfvec{B}(-a,-a) = -\frac{\mu_o}{4\pi} \frac{qv_o}{(8\sqrt{2})a^2}\khat$$

Check: Using the right--hand rule for cross products on $\bfvec{v}\times\hat{\mathbf{r}}$ confirms that the result is into the page.
\else
\vskip 288pt
\fi

\section{Problem II}

If $q$ is at $(x,y)=(a,0)$ and has a velocity of $\bfvec{v}=v_o\jhat$, find $\bfvec{B}$ vector at $(x,y)=(a,a)$.

\ifsolutions
\textbf{Answer}: $\bfvec{B}(a,a)=0$ (From a diagram, $\bfvec{v}$ and $\hat{\mathbf{r}}$ are parallel, so their cross product is zero.)
\else
\vskip 144pt
\fi

\ifsolutions

\else

\newpage
\fi

\section{Problem III}

If $q$ is at $(x,y)=(a,2a)$ and has a velocity of $\bfvec{v}=v_o\jhat$, find $\bfvec{B}$ at $(x,y)=(-a,-a)$.

\ifsolutions
\textbf{Answer}: 
$$\bfvec{B}(-a,-a)= \frac{\mu_o}{4\pi} \frac{2qv_o\khat}{13\sqrt{13}a^2}$$
\else
\vskip 216pt
\fi

\section{Problem IV}

If $q$ is at the position $(x_o,y_o)$ and has a velocity of $\bfvec{v}=v_x\ihat+v_y\jhat$, 

$$\bfvec{B}(x,y)= \frac{\mu_o}{4\pi} \frac{q}{r^3} \big[v_x(y-y_o) - v_y(x-x_o)\big]\khat$$

where

$$r=\sqrt{(x-x_o)^2+(y-y_o)^2}$$

\begin{enumerate}

  \item Explain why $\bfvec{B}$ only has a $\khat$ component.

        \ifsolutions
        \textbf{Answer}: The $\bfvec{r}$ and $\bfvec{B}$ vectors are in the $x$--$y$ plane, and the result of a cross--product is a vector that is perpendicular to the plane to the two crossed vectors.
        \else
        \vskip 36pt
        \fi

  \item Use this formula to find $\bfvec{B}$ for the example problem in section 2.

        \ifsolutions
        \textbf{Answer}: In the example problem, $q$ is at $(-a,-a)$ and has a velocity of $\bfvec{v}=v_o\ihat$ and we want to know $\bfvec{B}$ at $(a,a)$. In terms of the variables for the given equation, the position of the charge is $(x_o,y_o)=(-a,-a)$, the location where where we want to know $\bfvec{B}$ is $(x,y)=(a,a)$, $v_x=v_o$, and $v_y=0$. Plugging these values into

        \vskip 12pt

        $$r=\sqrt{(x-x_o)^2+(y-y_o)^2}$$

        \vskip 12pt

        $$
        \bfvec{B}(x,y)= \frac{\mu_o}{4\pi} \frac{q}{r^3} \big[v_x(y-y_o) - v_y(x-x_o)\big]\khat
        $$

        \vskip 12pt

        gives

        \vskip 12pt

        $$r=\sqrt{(a--a)^2+(a--a)^2}=\sqrt{8}a$$

        \vskip 12pt

        and

        \vskip 12pt

        $$
        \bfvec{B}(a,a)= \frac{\mu_o}{4\pi} \frac{q}{(\sqrt{8}a)^3} v_o(a--a)\khat
        $$

        \vskip 12pt

        Simplification gives the same result found in the example:

        \vskip 12pt

        $$
        \bfvec{B}(a,a) = \frac{\mu_o}{4\pi} \frac{qv_o}{(8\sqrt{2})a^2}\khat
        $$
        \else
        \vskip 96pt
        \fi

  \item Derive this formula.

        \ifsolutions
        \textbf{Answer}: 
        The vector from the position of $q$, $(x_o,y_o)$, to the point where we want to know the field, $(x,y)$, is

        $\bfvec{r} = (x-x_o)\ihat + (y-y_o)\jhat$, so $r=\sqrt{(x-x_o)^2+(y-y_o)^2}$.

        Using this with $\rhat=\bfvec{r}/r$ and $\bfvec{v}=v_x\ihat+v_y\jhat$ in

        \vskip 12pt

        $$
        \bfvec{B} = \frac{\mu_o}{4\pi}\frac{q\bfvec{v}\times\hat{\mathbf{r}}}{r^2}
        $$

        gives

        $$
        \bfvec{B}(x,y) = \frac{\mu_o}{4\pi}\frac{q (v_x\ihat+v_y\jhat)\times\ds\frac{(x-x_o)\ihat + (y-y_o)\jhat}{r}}{r^2}
        $$

        or

        $$
        \bfvec{B}(x,y) = \frac{\mu_o}{4\pi}\frac{q}{r^3}(v_x\ihat+v_y\jhat)\times \big[(x-x_o)\ihat + (y-y_o)\jhat\big]
        $$

        Using the Multiply through method for cross--products (and dropping the terms involving $\ihat\times\ihat$ and $\jhat\times\jhat$) gives

        $$
        \bfvec{B}(x,y) = \frac{\mu_o}{4\pi}\frac{q}{r^3}\big[v_x\ihat\times (y-y_o)\jhat + v_y\jhat\times (x-x_o)\ihat\big]
        $$

        Evaluation of the cross--products gives

        $$
        \bfvec{B}(x,y)= \frac{\mu_o}{4\pi} \frac{1}{r^3} \big[v_x(y-y_o) - v_y(x-x_o)\big]\khat
        $$
        \else

        \fi

\end{enumerate}

\end{document}